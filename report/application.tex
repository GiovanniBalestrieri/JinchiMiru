\section{The Application}

\subsection{Module Flow}
The Jena ontology API is a Java programming toolkit. Jena's ontology support is limited to ontology formalisms built on top of RDF.

RDFS is the weakest ontology language supported by Jena. With RDFS it is possible to build a simple hierarchy of concepts, and a hierarchy of properties. 
There are various different ontology languages available for representing ontology information on the semantic web. They range from the most expressive, OWL Full, through to the weakest, RDFS.

The ontology language used in this project is the OWL FULL. OWL language allows properties to be denoted as transitive, symmetric or functional, and allows one property to be declared to be the inverse of another.

One of the key benefits of building an ontology-based application is using a reasoner to derive additional truths about the concepts you are modelling. Jena includes support for a variety of reasoners through the inference API.

A common feature of Jena reasoners is that they create a new RDF model which appears to contain the triples that are derived from reasoning as well as the triples that were asserted in the base model. The ontology API can query an extended inference model and extract information not explicitly given.

\subsection{Use Cases}


\subsection{Exporting Ontology}


\subsection{Future works}

The following points assume that the robot is equipped with a spoken command recognition

Lu4r

PointCloud