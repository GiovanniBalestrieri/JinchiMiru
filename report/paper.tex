\documentclass[a4paper, 12pt]{article}
\usepackage[latin1]{inputenc}
\usepackage[english]{babel}
\usepackage[T1]{fontenc}
\setcounter{section}{0}
\setcounter{figure}{0}
\usepackage{graphicx}
\usepackage{float}
\usepackage[centertags]{amsmath}
\usepackage{amsfonts}
\usepackage{amsthm}
\usepackage{newlfont}

\usepackage{footnote}

\usepackage{fancyhdr}
\usepackage{tesisty}
\usepackage{enumitem}
\setlist[itemize]{noitemsep, leftmargin=10pt}
\usepackage{amssymb}
\usepackage{listings}
\usepackage{hyperref} 
\usepackage{lipsum}
\usepackage[framed,numbered,autolinebreaks,useliterate]{mcode}
\lstset{language=Matlab, basicstyle=\scriptsize\ttfamily, frame=single}
%\lstloadlanguages{Matlab}
%\usepackage{rotating}

%-------------------------------
% DEFINIZIONE DEGLI ENVIRONMENT
%-------------------------------

\newtheorem{obs}{Osservazione}[section]
\newenvironment{oss}
    {\begin{obs}\begin{normalfont}}
    {\hfill $\square \!\!\!\!\checkmark$ \end{normalfont}\end{obs}}

\newtheorem{pro}{Problema}[section]
\newenvironment{prob}
    {\begin{pro}\begin{normalfont}}
    {\hfill $\spadesuit$ \end{normalfont}\end{pro}}

\newtheorem{teor}{Teorema}[section]
\newenvironment{teorema}
    {\begin{teor}\textit }
    {\hfill  \end{teor}}

\newtheorem{defn}{Definizione}[section]
\newenvironment{de}
    {\begin{defn}\begin{normalfont}}
    {\hfill $\clubsuit$ \end{normalfont}\end{defn}}

%-----------------------------
% CONFIGURAZIONE DELLA PAGINA
%-----------------------------

\hfuzz2pt % Don't bother to report over-full boxes if over-edge is < 2pt

\fancypagestyle{plain}{
\fancyhead{}\renewcommand{\headrulewidth}{0pt} } \pagestyle{fancy}
%\renewcommand{\chaptermark}[1]{\markboth{\small CAP. \thechapter \textit{ #1}} {} }
\renewcommand{\sectionmark}[1]{\markright{\small  \thesection \textit{ #1}} {} }
\voffset=-20pt    % distanza tra il limite superiore del foglio e l'intestazione
\headsep=40pt     % distanza  l'intestazione ed il testo del corpo
\hoffset=0 pt     % misura equivalente al margine sinistro
\textheight=620pt % altezza del corpo del testo
\textwidth=435pt  % larghezza del corpo del testo
\footskip=40pt    % distanza tra il testo del corpo ed il pie' di pagina
\fancyhead{}      % cancella qualsiasi impostazione per l'intestazione
\fancyfoot{}      % cancella qualsiasi impostazione per il pie' di pagina
\headwidth=435pt  % larghezza del'intestazione e del pie' di pagina
\fancyhead[R]{}%{\rightmark} \fancyfoot[L]{\leftmark}
\fancyfoot[R]{}%{\thepage}
\renewcommand{\headrulewidth}{0.3pt}   % spessore della linea dell'intestazione
\renewcommand{\footrulewidth}{0.3pt}   % spessore della linea del pi�di pagina

\numberwithin{equation}{section}
\renewcommand{\theequation}{\thesection.\arabic{equation}}
%--------------------------
% MODIFICARE DA QUI IN POI
%--------------------------

\begin{document}
\corso{Artificial Intelligence} \titoloTesi{} \anno{2015/2016}

\baselineskip=25pt

\intestazione

%------------------------------------------------
% INTRODUZIONE E RINGRAZIAMENTI (NON MODIFICARE)
%------------------------------------------------

\fancypagestyle{plain}{
\fancyhead\renewcommand{\headrulewidth}{0pt} } \pagestyle{fancy}
%\renewcommand{\chaptermark}[1]{\markboth{\small Cap. \thechapter \textit{ #1}} {} }
\renewcommand{\sectionmark}[1]{\markright{\small  \S \thesection \textit{ #1}} {} }
\voffset=-20pt                         % distanza tra il limite superiore del foglio e l'intestazione
\headsep=40pt                          % distanza  l'intestazione ed il testo del corpo
\hoffset=0pt                           % misura equivalente al margine sinistro
\textheight=620pt                      % altezza del corpo del testo
\textwidth=435pt                       % larghezza del corpo del testo
\footskip=40pt                         % distanza tra il testo del corpo ed il pie' di pagina
\fancyhead{}                           % cancella qualsiasi impostazione per l'intestazione
\fancyfoot{}                           % cancella qualsiasi impostazione per il pie' di pagina
\headwidth=435pt                       % larghezza del'intestazione e del pie' di pagina
\fancyhead[R]{\rightmark} \fancyfoot[L]{\leftmark}
\fancyfoot[R]{\thepage}
\renewcommand{\headrulewidth}{0.3pt}   % spessore della linea dell'intestazione
\renewcommand{\footrulewidth}{0.3pt}   % spessore della linea del pié di pagina

\pagenumbering{Roman} \tableofcontents
\newpage

\pagenumbering{arabic}

\fancyhead[R]{Introduzione} \fancyfoot[L]{Introduzione}
\fancyfoot[R]{\thepage}


\fancyhf{} %elimina header/footer vecchi


\fancyhead[R]{\rightmark} \fancyhead[L]{\leftmark}
\fancyfoot[R]{\thepage}


%-----------------------
% DEFINIZIONE VARIABILI
%-----------------------

\newcommand{\figura}{figura}

%---------------------
% INCLUSIONE CAPITOLI
%---------------------

\begin{abstract}
This paper is an academic final report for the Artificial Intelligence course. The goal of the project is to implement a Ros module for building high-level representations of the environment that embody both metric and symbolic knowledge about it. A key issue in the interaction with robots is to establish a proper relationship between the symbols used in the representation and the corresponding elements of the operational environment.
\end{abstract}	

\section{Introduction}

Robotics is in an exciting stage and human machine interaction is being intensively studied.
Robots are expected to get closely involved into human life as they are being marketed for commercial applications such as telepresence, service or entertainment. However, although they are expected to become consumer products, there is still a gap in terms of user expectations and robot functionalities. A key limiting factor is the lack of awareness of the robot on the operational environment.
This project investigates several strategies to integrate a knowledge base in the ROS framework. The implemented system provides knowledge processing capabilities that combine knowledge representation and reasoning methods to manipulate and interact with physical objects of the operational milieu through an \textit{API system} and a graphical interface.\\

\subsection{Objectives}
\label{sec:objectives}

\begin{enumerate}
\item Integrate a knowledge base in ROS,
\item Create a node to Parse owl files in ROS environment,
\item Consistency check of ontology,
\item Reasoner invocation,
\item Make the ROS environment aware of object's instances,
\item Graphical representation of the objects in the scene,
\item Insert new instance of object in the scene and check for consistency,
\item Delete instance and check for consistency,
\item List instances,
\item Graphical Scene configuration from Gazebo and automatic Abox update
\item Export current Abox in owl or different formats.
\end{enumerate}

\subsection{Project Schedule}
A public git repository is available at the following \href{https://github.com/GiovanniBalestrieri/JinchiMiru}{Github page}\footnote{https://github.com/GiovanniBalestrieri/JinchiMiru} and a step by step guide has been published at www.userk.co.uk\footnote{More further information visit http://userk.co.uk/handling-ontologies-with-ros/}.
\small
\noindent 
\begin{center}
\makesavenoteenv{tabular}
    \begin{tabular}{ | l | p{4cm} | p{4cm} | p{5cm} |}
    \hline
    Week & General Task & Documentation & Implementation \\ \hline
    1 &  \begin{itemize} 
    \item Perform a literature review on the previous HuRIC
     publications
     \end{itemize} & 
	\begin{itemize}
     \item  Knowledge representation and Reasoning \cite{bib11}
     \item  Huric papers
      \cite{bib1},\cite{bib2},\cite{bib3},\cite{bib4}, \cite{bib5}
	\end{itemize} & \\ \hline
    2 & \begin{itemize}
     \item Literature review of ROS compatible triple store
     \item ROS compatible simulation environments
	\end{itemize}     
     & \begin{itemize}
     \item ROS Documentation \cite{bib6}, \cite{bib7}
     \item ROS compatible simulation environments \cite{bib8}, \cite{bib9}
	\end{itemize}   & 
	\begin{itemize} 
	\item Set up ROS environment.
	\item Brainstorm Software Architecture
	\end{itemize}\\ \hline
    3 & \begin{itemize}
     \item Literature review of RDFlib based papers
     \item Test of Gazebo Simulator
     \end{itemize}  & 
     \begin{itemize}
     \item Full Training session on Gazebo Simulator \cite{bib10} 
     \item Python library RDFLib test 
     \end{itemize} & 
     \begin{itemize}
     \item Test Json Parser node
     \item Populate a scene from json files. 
     \end{itemize} \\ \hline
    4 & \begin{itemize}
     \item Kinect pointcloud literature review
     \item Object recognition papers review
     \end{itemize}  & 
     \begin{itemize}
     \item PointCloudLibrary documentation \cite{bib16}
     \item Python SciPy library classifier documentation
     \end{itemize}
      & \begin{itemize}
     \item Parse test KnowledgeBase owl file with RDFLib
     \item Clear scene script in Gazebo.
     \end{itemize}  \\
    \hline
    5 & \begin{itemize}
     \item Integrate classification algorithms
     \item Optimize python code and ROS environment.
     \end{itemize} 
     & \begin{itemize}
     \item ROS + PCL integration
     \item RDF library deep inspection
     \end{itemize} 
      & \begin{itemize}
     \item Analyze pointcloud from kinect
     \item Scene analysis, plane segmentation
     \item Object recognition, vote classifier
     \end{itemize}  \\
    \hline
    6 & \begin{itemize}
     \item Owl Reasoner integration
     \item Test RDFlib and Apache Jena
     \end{itemize} 
     & \begin{itemize}
     \item RDF lib documentation
     \item Apache Jena Fuseki Documentation \cite{bib15}
     \end{itemize} 
      & \begin{itemize}
     \item Create init node
     \item Spawn models script
     \item Remove models script
     \item Apache Jena basic setup
    \end{itemize}  \\
    \hline
    \end{tabular}
\end{center}


\begin{center}
\makesavenoteenv{tabular}
    \begin{tabular}{ | l | p{4cm} | p{4cm} | p{5cm} |}
    \hline
    Week & General Task & Documentation & Implementation \\ \hline
    7 & \begin{itemize}
     \item Consistency check
     \item RosJava integration
     \item Jena Ontology Api
     \end{itemize} 
     & \begin{itemize}
     \item RosJava guidelines \cite{bib12}
     \item Apache Jena Ontology Api \cite{bib13}
     \end{itemize} 
      & \begin{itemize}
     \item Integrate RosJava in Ros.
     \item Follow Jena Ontology Api tutorial. Basic rdf manipulation
    \end{itemize}  \\
    \hline
    8 & \begin{itemize}
     \item Consistency check
     \end{itemize} 
     & \begin{itemize}
     \item RosJava guidelines \cite{bib12}
     \item Jena Reasoner Documentation \cite{bib14}
     \end{itemize} 
      & \begin{itemize}
     \item Implement consistency check with Jena.
     \item Integrate Jena libraries in ROS
     \item Reasoner invocation in ROS.
    \end{itemize}  \\
    \hline
    9 & \begin{itemize}
     \item Owl A-box extraction
     \item Specialize Reasoner on Tbox
     \end{itemize} 
     & \begin{itemize}
     \item RDFlib export documentation     
     \item Jena Reasoner documentation     
     \end{itemize}
      & \begin{itemize}
     \item Implement A-box generator Jena
     \item Retrieve information about instances     
    \end{itemize}  \\
    \hline
    10 & \begin{itemize}
     \item Get info about model in Gazebo 
     \item Specialize Reasoner on Tbox
     \end{itemize} 
     & \begin{itemize}
     \item Programming Robots with Ros \cite{bib9}      
     \item Gazebo Documentation \cite{bib10}    
     \end{itemize}
      & \begin{itemize}
     \item Subscribe to gazebo modelStates in ROSJava
     \item Call spawn model service from ROSJava
    \end{itemize}  \\
    \hline
    11 & \begin{itemize}
     \item SPARQL queries Add, Delete, Update, GET Instance
     \item Coordinator Node Definition
     \item Interface between Semantic Map and Gazebo Node
     \end{itemize} 
     & \begin{itemize}
     \item Jena Ontology Api Documentation \cite{bib13}      
     \item Gazebo Documentation \cite{bib10}    
     \end{itemize}
      & \begin{itemize}
     \item Jena RDF graph manipulation 
     \item Jena SPARQL Delete and Add queries 
     \item Jena SPARQL GetInstance queries 
    \end{itemize}  \\
    \hline
    
    \end{tabular}
\end{center}

\begin{center}
\makesavenoteenv{tabular}
    \begin{tabular}{ | l | p{4cm} | p{4cm} | p{5cm} |}
    \hline
    Week & General Task & Documentation & Implementation \\ \hline
    12 & \begin{itemize}
     \item Test Case validation
     \item Use Case validation
     \item Application Demo
     \end{itemize} &
      & \begin{itemize}
     \item Testing of real world scenario
     \item Bug fixing 
     \item Extern Node definition
    \end{itemize}  \\
    \hline
    \end{tabular}
\end{center}
 



\section{The Ontology}
It is expected that mobile robots undertake various tasks not only in the industrial fields such as manufacturing plants and construction sites, but also in the environment we live in.

\subsection{Tbox}
In this project a generic home domain model has been taken into account. 

\begin{figure}[H]
\centering
\includegraphics[width=0.3\textwidth]{imgs/ontology.png}
\label{fig:ontologyThings}
\caption{Physical Things}
\end{figure}

The furniture class describes several objects of a generic home environment a robot can interact with.

\begin{figure}[H]
\centering
\includegraphics[width=0.8\textwidth]{imgs/ontology1.png}
\label{fig:furniture}
\caption{furniture subclasses}
\end{figure}

\subsection*{Properties}
OWL distinguishes between two main categories of properties that an ontology builder may want to define:

\begin{itemize}
	\item Object properties link individuals to individuals.
	\item Datatype properties link individuals to data values.
\end{itemize}

An object property is defined as an instance of the built-in OWL class owl:ObjectProperty. A datatype property is defined as an instance of the built-in OWL class owl:DatatypeProperty. Both owl:ObjectProperty and owl:DatatypeProperty are subclasses of the RDF class rdf:Property.\\

\subsubsection*{Properties}
The datatypes involved are shown in the Figure ~\ref{fig:datatypes} and ~\ref{fig:datatypesProtege}

\begin{figure}[H]
\centering
\includegraphics[width=0.6\textwidth]{imgs/datatypeProtege.png}
\label{fig:datatypesProtege}
\caption{Datatypes Visual Prot\'eg\'e}
\end{figure}

\begin{figure}[H]
\centering
\includegraphics[width=0.6\textwidth]{imgs/Datatype.png}
\label{fig:datatypes}
\caption{Datatypes}
\end{figure}

\textit{Object Properties}

The following Figure \ref{fig:propClassTree} shows the class tree diagram of the ObjectProperties involved in the project.
\begin{figure}[H]
\centering
\includegraphics[width=0.6\textwidth]{imgs/propClassTree.png}
\label{fig:propClassTree}
\caption{Object Properties Class Tree}
\end{figure}

As an example, consider the following set of owl statements about the ObjectProperty hasPosition. This property is of the type IrreflexivePropery and is a subProperty of SpacialProperty.

\begin{lstlisting}
<owl:ObjectProperty rdf:about="sm#hasPosition">
    <rdfs:subPropertyOf rdf:resource="sm#hasSpatialProperty"/>
    <rdf:type rdf:resource="http://www.w3.org/2002/07/owl#IrreflexiveProperty"/>
</owl:ObjectProperty>
\end{lstlisting}    

Let us consider another Property involved in this project.

\begin{lstlisting}
<owl:Class rdf:about="sm#Coordinates">
    <rdfs:subClassOf rdf:resource="sm#Position"/>
    <rdfs:subClassOf>
       <owl:Restriction>
           <owl:onProperty rdf:resource="sm#float_coordinates_z"/>
           <owl:someValuesFrom rdf:resource="http://www.w3.org/2001/XMLSchema#float"/>
           </owl:Restriction>
     </rdfs:subClassOf>
     <rdfs:subClassOf>
        <owl:Restriction>
           <owl:onProperty rdf:resource="sm#float_coordinates_x"/>
           <owl:qualifiedCardinality rdf:datatype="http://www.w3.org/2001/XMLSchema#nonNegativeInteger">1
           </owl:qualifiedCardinality>
           <owl:onDataRange rdf:resource="http://www.w3.org/2001/XMLSchema#float"/>
           </owl:Restriction>
     </rdfs:subClassOf>
     <rdfs:subClassOf>
        <owl:Restriction>
           <owl:onProperty rdf:resource="sm#float_coordinates_y"/>
           <owl:qualifiedCardinality rdf:datatype="http://www.w3.org/2001/XMLSchema#nonNegativeInteger">1
			</owl:qualifiedCardinality>
           <owl:onDataRange rdf:resource="http://www.w3.org/2001/XMLSchema#float"/>
       </owl:Restriction>
   </rdfs:subClassOf>
</owl:Class>   
\end{lstlisting}

The class Coordinates is a subclass of the class Position and of three anonymous classes. A property restriction describes an anonymous class, namely a class of all individuals that satisfy the restriction. \\\
The first restriction is a Value contraint linked (using owl:onProperty) the Property float\_coordinates\_z  to a class of all individuals for which at least one value of the property concerned is an instance of a data value in the data range.\\
The second and third restrictions are Cardinality contraints linked to the Property float\_coordinates\_x and float\_coordinates\_y.


\subsection{Abox}
The collection of individual are stored in a separate file called semantic\_mapping, the Abox. The demo supports operations on four classes of instances since the 3D environment requires tridimentional models of the object to be represented. This constrain could be relaxed by adding an exhaustive collection of 3D models and by associating them to the corresponding classes.\\

\subsubsection*{An instance of the class Chair}
Individuals are defined with individual axioms called ``facts''. These facts are statements indicating class membership of individuals and property values of individuals. As an example, consider the following set of statements about an instance of the class Chair:

\begin{lstlisting}    
<NamedIndividual rdf:about="sm#chair1">
        <rdf:type rdf:resource="&semantic_mapping_domain_model;Chair"/>
        <semantic_mapping_domain_model:hasPosition rdf:resource="sm#chair1_coordinates"/>
        <semantic_mapping_domain_model:hasSize rdf:resource="sm#chair1_size"/>
        <semantic_mapping_domain_model:hasAlternativeReference rdf:resource="sm#chair_alternative_reference_1"/>
        <semantic_mapping_domain_model:hasAlternativeReference   rdf:resource="sm#chair_alternative_reference_2"/>
        <semantic_mapping_domain_model:hasAlternativeReference  rdf:resource="sm#chair_alternative_reference_3"/>
        <semantic_mapping_domain_model:hasPreferredReference rdf:resource="sm#chair_preferred_reference"/>
    </NamedIndividual>
\end{lstlisting}

This example includes a number of facts about the individual chair1, an instance of the class Chair. The chair has three alternative references and one preferred lexical reference. These properties link a chair to a typed literal with the XML Schema datatype date. The XML schema document on datatypes contains the relevant information about syntax and semantics of this datatype. The property hasPosition and hasSize link the chair to instances of the type Coordinates and Dimensions.

The following figure shows the same information on Prot\'eg\'e:

\begin{figure}[H]
\centering
\includegraphics[width=0.6\textwidth]{imgs/chair1.png}
\label{fig:datatypes}
\caption{Chair1}
\end{figure}

\textbf{Properties}

The following example shows AlternativeReference instance property.
\begin{lstlisting}    
<NamedIndividual rdf:about="sm#chair_alternative_reference_1">
        <rdf:type rdf:resource="&semantic_mapping_domain_model;AlternativeReference"/>
        <semantic_mapping_domain_model:lexicalReference rdf:datatype="&xsd;string">chair
        </semantic_mapping_domain_model:lexicalReference>
    </NamedIndividual>
\end{lstlisting}


\begin{figure}[H]
\centering
\includegraphics[width=0.6\textwidth]{imgs/refChair1.png}
\label{fig:datatypes}
\caption{Alternative Reference 1}
\end{figure}

The following example shows the instance of the 3Three\_Dim\_Size property associated to the individual chair1
\begin{lstlisting}    
<NamedIndividual rdf:about="sm#chair1_coordinates">
        <rdf:type rdf:resource="&semantic_mapping_domain_model;Coordinates"/>
        <semantic_mapping_domain_model:float_coordinates_z rdf:datatype="&xsd;float">0.0
        </semantic_mapping_domain_model:float_coordinates_z>
        <semantic_mapping_domain_model:float_coordinates_y rdf:datatype="&xsd;float">0.0
        </semantic_mapping_domain_model:float_coordinates_y>
        <semantic_mapping_domain_model:float_coordinates_x rdf:datatype="&xsd;float">1.0
        </semantic_mapping_domain_model:float_coordinates_x>
    </NamedIndividual>
\end{lstlisting}



\begin{figure}[H]
\centering
\includegraphics[width=0.6\textwidth]{imgs/coordchair1.png}
\label{fig:datatypes}
\caption{Coordinates chair 1}
\end{figure}

\include{technologies}
\section{The Architecture}

This project has been implemented using ROS framework and it is a modular application that enables other nodes to query and interact with the knowledge base. Four nodes are involved in this application and the communication between them relies on an inter-process protocol handled by ROS.

\subsection{System description}

The knowledge base is loaded from the SemanticMapInterface which is a Java node based on the Jena Ontology API. This process allows requesting nodes to access and manipulate the ontology using a predefined set of operations.

\begin{figure}[H]
\centering
\includegraphics[width=0.8\textwidth]{imgs/semantic.jpg}
\label{fig:actions}
\caption{SemanticMapInterface Node}
\end{figure}

The exposed operations include:

\begin{itemize}
\item Loading the Terminology (TBox) and Assertion Box (ABox) into memory,
\item Listing instances, their spacial properties and their preferred lexical reference,
\item Adding a new entity of a particular class, with the spacial and lexical properties specified as arguments,
\item Updating properties of active entities,
\item Removing active entities,
\item Invoking OWL-FULL reasoner and performing inference operation given a domain model, 
\item Exporting in OWL format the list of instances present or the augmented ABox with derived properties.

\end{itemize}

The main component is the Coordinator node. This python script is an indipendent module that performs several actions and communicates with other nodes in the system. Below are listed the supported actions:

\begin{itemize}
\item Handling high level requests from other nodes such as visualizing or exporting the instances of objects present the ontology.
\item Request operations to SemanticMapInterface or to Gazebo simulator. 
\item 
\end{itemize}



\begin{figure}[H]
\centering
\includegraphics[width=0.8\textwidth]{imgs/architecture.jpg}
\label{fig:actions}
\caption{System Architecture}
\end{figure}


\subsection{Test case}
\section{The Application}

The application consists of a number of indipendent nodes that comprise a graph. Each node sends and receives information to and from other nodes of the graph using \textit{Topics}.
The communication between nodes strongly relies on a publish/subscribe mechanism useful to  exchange data in a distributed system. 


\subsection{Use cases}

\subsubsection{Initialization}
The application is launched by executing the init.launch file. The Roslaunch tool allows to launch multiple ROS nodes as well as set several parameters for the simulation environment. the initialization process brings up the master node, roscore, the coordinator, the semanticMapInterface and Gazebo.\\
\\
Once initialized, the SemanticMapInterface loads from a specified absolute path the Terminology Box and the Assertions Box as Ontology Models.

\begin{lstlisting}[language=Java]
/**
 * Importing Tbox
 */
 tbox = ModelFactory.createOntologyModel( OntModelSpec.OWL_MEM );
 OntDocumentManager dm_tbox = tbox.getDocumentManager();
 dm_tbox.addAltEntry(SOURCE+TBOX_FILE,"file:"+TBOX_FILE);
 tbox.read(SOURCE+TBOX_FILE,"RDF/XML");

 /**
 * Importing Abox
 */
 abox = ModelFactory.createOntologyModel( OntModelSpec.OWL_MEM);
 OntDocumentManager dma = abox.getDocumentManager();
 dma.addAltEntry( SOURCE + ABOX_FILE , "file:" + ABOX_FILE);
 abox.read(SOURCE + ABOX_FILE,"RDF/XML");
\end{lstlisting}

The reasoner API supports the notion of specializing a reasoner by binding it to a set of schema or ontology data using the bindSchema call. The specialized reasoner can then be attached to different sets of instance data using bind calls. It is worth noting that in this project the schema (TBox) and instance (ABox) data were saved in two separate files.

\begin{lstlisting}[language=Java]
Reasoner reasoner = ReasonerRegistry.getOWLReasoner();
reasoner = reasoner.bindSchema(tbox);
OntModelSpec ontModelSpec=OntModelSpec.OWL_MEM_MICRO_RULE_INF;
ontModelSpec.setReasoner(reasoner);
InfModel infmodel = ModelFactory.createInfModel(reasoner,abox);
\end{lstlisting}
This is equivalent to an Ontology Model with Reasoner capabilities
specialized on the ABox.
\begin{lstlisting}[language=Java]
infModel = ModelFactory.createOntologyModel( OntModelSpec.OWL_MEM_MICRO_RULE_INF, abox);
\end{lstlisting}

Typically the ontology languages used with the semantic web allow constraints to be expressed, the validation interface is used to detect when such constraints are violated by some data set. To test for inconsistencies with a data set using a reasoner the InfModel.validate() interface. This performs a global check across the schema and instance data looking for inconsistencies. 


\begin{lstlisting}[language=Java]
/** 
 * Consistency Check. Returns true if passed
 */
 private static boolean performConsistencyCheckWith(InfModel inf) {
 boolean res = false; 

 ValidityReport validity = inf.validate();
 if (validity.isValid()) {
	System.out.println(""+ NODE_NAME + "\tConsistency Check:\n Passed\n");
	res = true;
 } else {
	System.out.println(""+ NODE_NAME + "\tConsistency Check:\n Conflicts\n");
	for (Iterator i = validity.getReports(); i.hasNext(); ) {
	    System.out.println(" - " + i.next());
	}
 }
 return res;
}
\end{lstlisting}

When the Coordinator is initialized it sends on topic called Bridge an init request. The request is captured by the SemanticMap and the $getAllInstances(OntModel)$ method is called. 


\begin{figure}[H]
\centering
\includegraphics[width=0.8\textwidth]{imgs/topics1.jpg}
\label{fig:actions}
\caption{Data exchange between Coordinator and SemanticMapInterface}
\end{figure}


A collection of instances is retrieved and saved in a HashMap. This method performs the following SPARQL query which returns all Furniture and Drink\footnote{Furniture and Drink are classes defined in the Terminology Box} entities.

\begin{lstlisting}[language=Java]
String queryString = "PREFIX rdf: <http://www.w3.org/1999/02/22-rdf-syntax-ns#>" +
				"prefix rdfs: <"+RDFS.getURI()+">\n" +
	    		"PREFIX xsd: <http://www.w3.org/2001/XMLSchema#> " +
	    		"PREFIX hasPosition: <" + NS + POSITION +"> " +
	    		"PREFIX hasRef: <" + NS + PREF_REF +"> " +
				"prefix semantic_mapping_domain_model: <" + DOMAIN_MODEL_NS + "#> \n"+
				"prefix semantic_mapping_1: <" + SEMANTIC_MAP_NS + "#> \n"+
	    		
	    		"PREFIX coordx: <" + NS + COORD_X +"> " +
	    		"PREFIX coordy: <" + NS + COORD_Y +"> " +
	    		"PREFIX coordz: <" + NS + COORD_Z +"> " +
	    		"PREFIX prefRef: <" + NS + LEXICAL +"> " +
	    		
	    		
	    		"SELECT DISTINCT ?uri ?class ?x ?y ?z ?lex "+
	    		"WHERE {" + 
	    			"{"+
		    		"?uri a ?class ." + 
		    		"?class rdfs:subClassOf semantic_mapping_domain_model:Furniture ."+
		    		"?uri hasPosition: ?pos ." + 
		    		"?uri hasRef: ?ref ." + 
		    		"?ref prefRef: ?lex ." + 
		    		"?pos coordx: ?x . " + 
		    		"?pos coordy: ?y . " + 
		    		"?pos coordz: ?z " +  "} UNION {"+
		    		"?uri a ?class ." + 
		    		"?class rdfs:subClassOf semantic_mapping_domain_model:Drink ." +
		    		"?uri hasPosition: ?pos ." + 
		    		"?uri hasRef: ?ref ." + 
		    		"?ref prefRef: ?lex ." + 
		    		"?pos coordx: ?x . " + 
		    		"?pos coordy: ?y . " + 
		    		"?pos coordz: ?z " +  "}"+
"}" ;
\end{lstlisting}
The resulting information are embedded in a message and sent on a topic called \textit{Huric\_jena}.

The Coordinator Node receives and parses the message. It then calls a special service that allows to create or deleted models dynamically in the simulation environment of Gazebo.


\subsubsection{Insertion of entities}
Instances of a given class can be dynamically inserted in the Assertion Box by publishing a predefined command on a topic called \textit{extern\_commands}. One node can create an instance by specifying its super class, its spacial positions and preferred lexical reference. Another faster way include the possibility to add a Chair in fixed position. These insertion methods are issued by the ExternInterface node, captured by the Coordinator and forwarded to the SemanticMapInterface.


\begin{figure}[H]
\centering
\includegraphics[width=0.8\textwidth]{imgs/topics2.jpg}
\label{fig:actions}
\caption{Data exchange between ExternInterface, Coordinator and SemanticMapInterface}
\end{figure}


The Java Node provides two methods to insert an instance of given type to the ontology model loaded in memory:
\begin{itemize}
\item Triple manipulation based
\item SPARQL based
\end{itemize}
Jena Ontology API provides a full set of methods to manipulate statements. The handleAddEntityRequest(ontModel,ontClass, uri, pose,lexicalReference) method is given below:

\begin{lstlisting}[language=Java]
/**
 * Handles AddEntityRequests from orchestrator using Jena API
 *	Creates a new instance, adds properties to it and perform consistency check
 * If test is passed, leaves the newly created instance, otherwise deletes it
 */
	private static boolean handleAddEntityRequest(OntModel abox, OntClass ontClass, String uriInstance, String posX,
			String posY, String posZ, String lexicalReference){
		boolean res = false;

	    OntClass prefrefClass = abox.getOntClass(NS+PREF_REF_CLASS);
	    OntClass coordinatesClass = abox.getOntClass(NS+COORDINATES_CLASS);

		String uriBase = "http://www.semanticweb.org/ontologies/2016/1/semantic_mapping_1#";
		// Create instance
	    //OntClass class1 = abox.getOntClass(NS+ontClass);
		
		if (ontClass == null)
			return false;
		else {
			// Create Individuals
			Individual i1 = abox.createIndividual(uriInstance,ontClass);
		    Individual prefRefInd = abox.createIndividual(uriBase+lexicalReference
		    +"_pref_ref",prefrefClass);
		    Individual coordinatesInd = abox.createIndividual(uriBase+lexicalReference
		    +"_pref_ref",coordinatesClass);
			
			
		    // Create Datatype Property
		    DatatypeProperty lexicalRef = abox.getDatatypeProperty(NS + LEXICAL);
			Literal ref = abox.createTypedLiteral(lexicalReference);
			Statement refStatement = abox.createStatement(prefRefInd, lexicalRef, ref);
			abox.add(refStatement);
			
			//  Bind pref reference individual to object individual
			ObjectProperty hasPrefRef = abox.getObjectProperty(NS+PREF_REF);
			Statement bindPrefRef = abox.createStatement(i1, hasPrefRef, prefRefInd);
			abox.add(bindPrefRef);
			

		    // Create Datatype Property for coordinate X
		    DatatypeProperty posx = abox.getDatatypeProperty(NS + COORD_X);
			Literal posXFloat = abox.createTypedLiteral(posX);
			Statement posXStatement = abox.createStatement(coordinatesInd, posx, posXFloat);
			abox.add(posXStatement);
		    // Create Datatype Property for coordinate X
		    DatatypeProperty posy = abox.getDatatypeProperty(NS + COORD_Y);
			Literal posYFloat = abox.createTypedLiteral(posY);
			Statement posYStatement = abox.createStatement(coordinatesInd, posy, posYFloat);
			abox.add(posYStatement);
		    // Create Datatype Property for coordinate X
		    DatatypeProperty posz = abox.getDatatypeProperty(NS + COORD_Z);
			Literal posZFloat = abox.createTypedLiteral(posZ);
			Statement posZStatement = abox.createStatement(coordinatesInd, posz, posZFloat);
			abox.add(posZStatement);
			

			//  Bind position individual to object individual
			ObjectProperty hasPosition = abox.getObjectProperty(NS+POSITION);
			Statement bindPose = abox.createStatement(i1, hasPosition, coordinatesInd);
			abox.add(bindPose);
			
			// Now perform consistency check

		    InfModel infModel = ModelFactory.createOntologyModel( OntModelSpec.OWL_MEM_MICRO_RULE_INF, abox);
			res = performConsistencyCheckWith(infModel);
		}
		
		return res;
}
\end{lstlisting}

The second method is called AddEntityRequest(abox,ontclass,uri,pose,lexicalReference) and it is based on a SPARQL query:

\begin{lstlisting}[language=Java]
/**
 * Handles AddEntityRequests using SPARQL
 *	Creates a new instance, adds properties to it and perform consistency check
 * If test is passed, leaves the newly created instance, otherwise deletes it
 */
 private static boolean AddEntityRequest(OntModel abox, OntClass ontClass, String uriInstance, String posX,
			String posY, String posZ, String lexicalReference){
 boolean res = true;
		
 String uri_Instance[] = uriInstance.split("#");
		
 if (ontClass == null) {
    res = false;
	System.out.print("\n"+ NODE_NAME + "\t[ Add Entity ] Class problem\n");
 } else {
			String queryString = "" + 
			"prefix rdfs: <"+ RDFS.getURI() +">\n" +
			"prefix rdf: <http://www.w3.org/1999/02/22-rdf-syntax-ns#> \n"+
			"PREFIX xsd: <http://www.w3.org/2001/XMLSchema#> "
			
			+ "prefix semantic_mapping_domain_model: <" + DOMAIN_MODEL_NS + "#> \n"
			+ "prefix semantic_mapping: <" + SEMANTIC_MAP_NS + "#> \n"
			+ "PREFIX class: <"+ ontClass.getURI() +">\n"
			
			
			+ "insert data { semantic_mapping:"+uri_Instance[1] +" rdf:type class: . "
			
			
			// Add hasPosition Irreflexive ObjectProperty 
			+ "semantic_mapping:" + uri_Instance[1] + " semantic_mapping_domain_model:hasPosition semantic_mapping:" + uri_Instance[1]
			+ "_coordinates . " 
			
			// Add Instance Coordinates 
			+ "semantic_mapping:" + uri_Instance[1]+ "_coordinates rdf:type semantic_mapping_domain_model:Coordinates . " 

			// Add datatype Property
			+ "semantic_mapping:" + uri_Instance[1] + "_coordinates semantic_mapping_domain_model:float_coordinates_x \""
			+ posX + "\"^^xsd:float . "
			+ "semantic_mapping:" + uri_Instance[1] + "_coordinates semantic_mapping_domain_model:float_coordinates_y \"" + posY
			+ "\"^^xsd:float . "
			+ "semantic_mapping:" + uri_Instance[1] + "_coordinates semantic_mapping_domain_model:float_coordinates_z \"" + posZ
			+ "\"^^xsd:float . "

			// Add hasPreferredReference ObjectProperty
			+ "semantic_mapping:" + uri_Instance[1] + " semantic_mapping_domain_model:hasPreferredReference semantic_mapping:" + uri_Instance[1]
			+ "_preferredReference . " 

			// Create Instance Lexical Reference
			+ "semantic_mapping:" + uri_Instance[1] + "_preferredReference rdf:type semantic_mapping_domain_model:PreferredReference . "

			+ "semantic_mapping:" + uri_Instance[1] + "_preferredReference semantic_mapping_domain_model:lexicalReference \"" + lexicalReference
			+ "\"^^xsd:string . " + "} \n ";
			
			UpdateAction.parseExecute(queryString, abox);
			
			// Check inconsistencies
		    InfModel infModel = ModelFactory.createOntologyModel( OntModelSpec.OWL_MEM_MICRO_RULE_INF, abox);
			res = performConsistencyCheckWith(infModel);
		}
	
		return res;
}
\end{lstlisting}

Once the instance has been inserted into the ontology model, the collection of new entities is once again sent on the \textit{huric\_jena} topic and captured by the coordinator which invokes the Gazebo service to render the 3D model of the newly created instances at the provided position in space.

\subsection{Exporting Ontology}


\subsection{Future works}

The following points assume that the robot is equipped with a spoken command recognition

Lu4r

PointCloud

% ELENCO DELLE FIGURE (OPZIONALE)
\addcontentsline{toc}{chapter}{Elenco delle figure}
\listoffigures


% BIBLIOGRAFIA
\addcontentsline{toc}{chapter}{References}
\begin{thebibliography}{11}
	
	\bibitem{bib1} E. Bastianelli, D. D. Bloisi, R. Capobianco, F. Cossu,
G. Gemignani, L. Iocchi, and D. Nardi,
            \emph{``On-line semantic mapping''} in 16th International Conference on Advanced
Robotics, Nov 2013.

	\bibitem{bib2} Emanuele Bastianelli, Danilo Croce, Roberto Basili, Daniele Nardi,
            \emph{``Using Semantic Maps for Robust Natural Language Interaction with Robots''}, DICII, 2 DII - University of Rome Tor Vergata, DIAG - Sapienza University of Rome - Rome, Italy.

	\bibitem{bib3} Emanuele Bastianelli, Giuseppe Castellucci, Danilo Croce, Luca Iocchi, Roberto Basili, Daniele Nardi,
            \emph{``HuRIC: a Human Robot Interaction Corpus''} 

	\bibitem{bib4} E. Bastianelli, Giuseppe Castellucci, Danilo Croce, Roberto Basili, Daniele Nardi,
            \emph{``Effective and Robust Natural Language Understanding
for Human -R obot Interaction''}, ECAI, 2014.
	
	\bibitem{bib5} E. Bastianelli, D. Bloisi, R. Capobianco, G. Gemignani, L. Iocchi, D. Nardi,
            \emph{``Knowledge Representation for Robots through
Human-Robot Interaction''}, Rome.

	\bibitem{bib6} Aaron Martinez, Enrique Fern\'andez,
            \emph{``Learning ROS for Robotics Programming''} A practical, instructive, and comprehensive guide to introduce yourself to ROS, the top-notch, leading robotics framework, 2013.
            
    \bibitem{bib7} Lentin Joseph,
            \emph{``Learning Robotics Using Python''} Design, simulate, program, and prototype an interactive autonomous mobile robot from scratch with the help of Python, ROS, and Open-CV!, 2015.
            
    \bibitem{bib8} Lentin Joseph,
            \emph{``Mastering ROS for Robotics Programming''}, 2015 Packt Publishing.
                  
    \bibitem{bib9} Morgan Quigley, Brian Gerkey and William D. Smart,\\
            \emph{``Programming Robots with ROS''}, 2016 O Reilly Media.
                     
    \bibitem{bib10} Open Source Robotics Foundation,\\
            \emph{``GazeboSim documentation''}, http://gazebosim.org/tutorials, 2016.
                       
    \bibitem{bib11} R. Brachman, H. Levesque,\\
            \emph{``Knowledge Representation and Reasoning''}.
            
    \bibitem{bib12} Open Source Robotics Foundation, Ros Documentation,\\
            \emph{``$http://wiki.ros.org/rosjava_build_tools/Tutorials/hydro$''}.
            
    \bibitem{bib13} Jena Ontology Api, \\
            \emph{``https://jena.apache.org/documentation/ontology/''}.
            
    \bibitem{bib14} Jena Reasoner and Inference documentation,\\
            \emph{``http://jena.apache.org/documentation/inference/index.html''}.
            
            
    \bibitem{bib15} Apache Jena Fuseki standalone server documentation,\\
            \emph{``$https://jena.apache.org/documentation/serving_data/$''}.
            

    \bibitem{bib16} Point Cloud Library Documentation, \\
            \emph{``http://pointclouds.org/documentation/tutorials/index.php''}.            
          
            
            
            
    

        
\end{thebibliography}
\end{document}
