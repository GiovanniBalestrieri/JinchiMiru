\chapter*{Appendice A\\ Identification.m}
\addcontentsline{toc}{chapter}{Appendice A - Identification.m}


\begin{lstlisting}

%% CARICAMENTO DATI
% 

stringa = {'dati_motore1/prova_1/','dati_motore1/prova_2/',
'dati_motore1/prova_3/','dati_motore1/prova_4/',
'dati_motore1/prova_5/','dati_motore1/prova_6/',
'dati_motore1/prova_7/','dati_motore1/prova_8/','dati_motore1/prova_9/'};

%I files delle misure (misure.txt) contegono un primo numero intero, 
%il tempo di campionamento in millisecondi, e poi due colonne di 
%numeri float separati da tabulazione semplice

Nfig = 1;
figure(Nfig)
Nfig = Nfig +1;
for i=1:length(stringa)
    
    %caricamento dati
    filep =fopen(strcat(char(stringa(i)),'measures.txt'),'r');
    Ts = 0.001*fscanf(filep, '%d %',1);
    measures = fscanf(filep, '%f %f',[2 inf]); 
    measures = measures';
    fclose(filep);
    
    %
    time(:,i) = Ts*[0:1:length(measures(:,1))-1];
    dati_input(:,i) = measures(:,1); 
    dati_output(:,i) = measures(:,2);
    subplot(length(stringa),1,i)
    plot(time(:,i),dati_input(:,i),'-r',time(:,i),dati_output(:,i),
    '-b','linewidth',1.5) 
    ylabel('$rad$','Interpreter','Latex')
end


 figure(Nfig)
 Nfig = Nfig +1;
 
tsim = time;
for i=1:length(stringa)
     ysim(:,i) = interp1(time(:,i)-time(1,i),dati_output(:,i)',tsim(:,i));
     usim(:,i) = interp1(time(:,i)-time(1,i),dati_input(:,i)',tsim(:,i));
     yfiltrata(:,i)=ysim(:,i);
     ufiltrata(:,i)=usim(:,i);
     subplot(length(stringa),1,i)
    plot(tsim(:,i),yfiltrata(:,i)/max(abs(yfiltrata(:,i))),'-r',tsim(:,i),
    ufiltrata(:,i)/max(abs(ufiltrata(:,i))),'-b','linewidth',1.5) 
    ylabel('$rad$','Interpreter','Latex')
end
xlabel('Time (sec)','Interpreter','Latex')
legend('output','input');
title('Filtered normalized data')

for i=1:length(stringa)
    for j=1:length(ufiltrata)*(2/3)-1
        ufiltmeta(j,i)=ufiltrata(j,i);
        yfiltmeta(j,i)=yfiltrata(j,i);
    end
end

%prepara i dati unendo le diverse prove sperimentali  
data = merge(iddata(yfiltrata(:,1),ufiltrata(:,1),Ts),...
              iddata(yfiltrata(:,2),ufiltrata(:,2),Ts),...
              iddata(yfiltrata(:,3),ufiltrata(:,3),Ts),...
              iddata(yfiltrata(:,4),ufiltrata(:,4),Ts),...
              iddata(yfiltrata(:,5),ufiltrata(:,5),Ts),...
              iddata(yfiltrata(:,6),ufiltrata(:,6),Ts),...
              iddata(yfiltrata(:,7),ufiltrata(:,7),Ts),...
              iddata(yfiltrata(:,8),ufiltrata(:,8),Ts),...
              iddata(yfiltrata(:,9),ufiltrata(:,9),Ts));

%prepara i dati unendo le diverse prove sperimentali 
datameta = merge(iddata(yfiltmeta(:,1),ufiltmeta(:,1),Ts),...
             iddata(yfiltmeta(:,2),ufiltmeta(:,2),Ts),...
             iddata(yfiltmeta(:,3),ufiltmeta(:,3),Ts),...
             iddata(yfiltmeta(:,4),ufiltmeta(:,4),Ts),...
             iddata(yfiltmeta(:,5),ufiltmeta(:,5),Ts),...
             iddata(yfiltmeta(:,6),ufiltmeta(:,6),Ts),...
             iddata(yfiltmeta(:,7),ufiltmeta(:,7),Ts),...
             iddata(yfiltmeta(:,8),ufiltmeta(:,8),Ts),...
             iddata(yfiltmeta(:,9),ufiltmeta(:,9),Ts));



%%%%%%%%%%%%%%%%%%%%%%%%%%%%%%%%%%%%%%%%%%%%%%
%PARAMETRI DA MODIFICARE 
           
na = 4%grado denominatore
nb = 1%grado numeratore
nc = 0 %e' il grado del numeratore che si associa al disturbo
nk = 0 %RITARDO in campioni 
 
nf = 0 %grado denominatore ingresso   , per  il metodo PEM      
nd = 0 %grado denominatore errore , per il metodo PEM


focuses = {'simulation','prediction'};
focus =  2; %seleziona

%tolleranza
idtol = 1e-8; %seleziona

%numero massimo di iterazioni
maxiter = 500; %seleziona

metodo = {'arx','armax','pem'};
scelta = 3; %seleziona

%
%FINE SCELTA PARAMETRI 
%%%%%%%%%%%%%%%%%%%%%%%%%%%%%%%%%%%%%%%%%%%%%%%%%


%%LANCIO ALGORITMI DI IDENTIFICAZIONE 
switch char(metodo(scelta))
    
    case 'arx'
    %A(q) y(t) = B(q) u(t-nk) + e(t)
    orders = [na nb nc]
    modello_totale = arx(datameta,orders,'focus',char(focuses(focus)),
    'tolerance',idtol,'maxiter',maxiter)
        
    case 'armax'
    orders = [na nb nc nk]
    modello_totale = armax(datameta,orders,'focus',char(focuses(focus)),
    'tolerance',idtol,'maxiter',maxiter)
        
    case 'pem'
    % A(q) y(t) = [B(q)/F(q)] u(t-nk) + [C(q)/D(q)] e(t)
    orders =[na nb nc nd nf nk]
    modello_totale = pem(datameta,orders,'focus',char(focuses(focus)),
    'tolerance',idtol,'maxiter',maxiter)
        
end


Fmodello = tf(modello_totale.b,modello_totale.a,Ts)
Nfig = Nfig +1;
figure(Nfig)
bode(Fmodello)%,[0.1:0.1:2*pi*1/Ts*1.1])
grid on
title('Identified model bode diagram ')


Nfig = Nfig +1;
figure(Nfig)
title('Exeperimental Vs Simulated ')
for i=1:length(stringa) 
    ymodello(:,i) = lsim(Fmodello,ufiltrata(:,i),tsim(:,i));
    subplot(length(stringa),1,i)
    plot(tsim(:,i),yfiltrata(:,i),'-b',tsim(:,i),ymodello(:,i),'-r',
    'linewidth',1.5) 
    ylabel('$rad$','Interpreter','Latex')
end
xlabel('Time (sec)','Interpreter','Latex')
legend('Experimental data','Estimated data')
%l2 = legend('Ref$_{new}$','Ref$_{old}$','Ref$_{experimental}$')
%set(l2,'Interpreter','Latex')

advice(modello_totale)

Nfig = Nfig +1;
figure(Nfig)
compare(data,modello_totale) %utilizza tutti i valori con data
title('Compare (it uses proper initial conditions for matchings)')
% 
% Nfig = Nfig +1;
% figure(Nfig)
% e = resid(data,modello_totale,'fr')
% me = arx(e,[10 10 0]);
% bode(me,[1:0.1:180],'sd',3,'fill')
% title('')
% grid on
% 
% e = resid(data,modello_totale,'corr')
% resid(datameta,modello_totale)
% title('Residui su met� dei dati sperimentali')
% Nfig = Nfig +1;
% figure(Nfig)
% resid(data,modello_totale)
% title('Residui su tutti i dati sperimentali')

%Fcontinuous = d2c(Fmodello)
Nfig = Nfig +1;
figure(Nfig)
rlocus(Fmodello)
\end{lstlisting}

\chapter*{Appendice B\\ Arduino: Controllo motore 1}
\addcontentsline{toc}{chapter}{Appendice B - Arduino: Controllo motore 1}

\begin{lstlisting}
//H BRIDGE L 298
int DeadZone=0;
int DeadZone2=0;
//First Motor
#define PWM1 5 
#define Enable1 4
#define Enable2 6
//Second Motor
#define PWM2 11 
#define Enable3 12
#define Enable4 10

//ENCODERS AMT
#define CountsXround 2048
//Encoder Motor 1
#define encoder1PinA 2 
#define encoder1PinB 7 

volatile int EncoderCounter1 = 0;
double alpha_past = 0;
double vel_M1[5];

//Encoder Motor 2
#define encoder2PinA 3 
#define encoder2PinB 8 

volatile int EncoderCounter2 = 0;
double beta_past = 0;
double vel_M2[5];

//volatile boolean PastA = 0;
//volatile boolean PastB = 0;

//CONTROL related parameters Motor 1
#define Ts 1000
#define Kp 10
#define ControlDeadzone 3 
// then let u=0.
int u2 = 0;
int u4 = 0; //per il secondo motore
int i=0,j=0;

//OTHERS
unsigned long oldtime = 0;
long time = 0;
long time1=0;
long time2=0;
long time3=0;
long t21=0;
long t212=0;

int encdes=0; //Used in loop
int encdes2=0;

int enc0=0;  //Used for dynamic DeadZone
int enc1=0;  //Used for dynamic DeadZone

int u2_1=0;
int u2_2=0;
int u2_3=0;
int e_1=0;
int e_2=0;
int e_3=0;
int e_4=0;
int e=0;
float u2_t=0;

int in=0;
int t=0;
int pos=0;

int u4_1=0;
int u4_2=0;
int e2_1=0;
int e2_2=0;
int e2=0;
float u4_t=0;

void setup()
{
  //Setting up the Signals on the Arduino Board
  //Signals for Motor 1
  pinMode(encoder1PinA, INPUT);
  pinMode(encoder1PinB, INPUT);
  pinMode(Enable1, OUTPUT);
  pinMode(Enable2, OUTPUT);
  //Signals for Motor 2
  pinMode(encoder2PinA, INPUT);
  pinMode(encoder2PinB, INPUT);
  pinMode(Enable3, OUTPUT);
  pinMode(Enable4, OUTPUT);
  //Attaching the interrupts
  //encoder1 pin on interrupt 0 (arduino's pin 2)
  attachInterrupt(0, doEncoder1, CHANGE);
  //encoder2 pin on interrupt 1 (arduino's pin 3)
  attachInterrupt(1, doEncoder2, CHANGE);
  /*
  digitalWrite(Enable1, HIGH);
  digitalWrite(Enable2, LOW);
  analogWrite(PWM1,DeadZone);
  delay(400); //powering the motor
  analogWrite(PWM1,0); //breaking the motor
  delay(500); //it should be quite now...
  */
  //Serial interface intializing
  //Serial.begin(115200);
  Serial.begin(9600);
  Serial.println("Inizio!!!");
  Serial.println(1,DEC);
  Serial.println(EncoderCounter1,DEC);
  //Serial.println(EncoderCounter2,DEC);
  
  
  //Dynamic DeadZone
  enc0=digitalRead(encoder1PinA);
  u2=40;
  digitalWrite(Enable1,LOW);
  digitalWrite(Enable2,HIGH);
  while(enc0==digitalRead(encoder1PinA))
  {
     u2++;
     analogWrite(PWM1, constrain(u2,0,255)); 
     delay(200);
  }
  DeadZone=u2;
  u2=0;
  
  enc1=digitalRead(encoder2PinA);
  u4=40;
  digitalWrite(Enable3,LOW);
  digitalWrite(Enable4,HIGH);
  while(enc1==digitalRead(encoder2PinA))
  {
     u4++;
     analogWrite(PWM2, constrain(u4,0,255)); 
     delay(200);
  }
  DeadZone2=u4;
  u4=0;
  
  Serial.println("DeadZone motore 1:");
  Serial.print(DeadZone);
  Serial.print("\n");
  Serial.println("DeadZone motore 2:");
  Serial.print(DeadZone2);
  Serial.print("\n");
  
  delay(500);
  oldtime=micros();
  time1=oldtime;
  //This Arduino is the Master and starts to send Data
}

void loop()
{ 
  if(micros()-oldtime>= Ts)
  {
    oldtime = micros();
      
    encdes=2900*3;
    encdes2=0;      
    
    e=encdes-EncoderCounter1;
    e2=encdes2-EncoderCounter2;
    
    if (e>20||e<-20)
    {
    //Identificazione 1
    u2_t=Kp*(e-1.35*e_1+0.38*e_2)-0.11*u2_1+0.0726*u2_2; 
    
    //Identificazione 3
    //u2_t=Kp*(e -1.801*e_1+0.8259*e_2)+1.144*u2_1 -0.5217*u2_2; 
    //u2_t=Kp*(e -1.801*e_1+0.8259*e_2)-0.45*u2_1; 
    //u2_t=Kp*(e-2.601*e_1+2.267*e_2-0.6607*e_3)-0.05*u2_1+0.18*u2_2;
    
    //Identificazione 4
    //u2_t=Kp*(e-1.222*e_1+0.495*e_2-0.1506*e_3)-0.4*u2_1-0.21*u2_2; 
    //u2_t=Kp*(e-1.548*e_1+0.5934*e_2)+0.3744*u2_1+0.1776*u2_2; 
    }
    else 
    {
    u2_t=0;
    u2_3=0;
    u2_2=0;
    u2_1=0;
    e_1=0;
    e_2=0;
    e_3=0;
    e_4=0;
    }
    u2=(int)u2_t;
    e_4=e_3;
    e_3=e_2;
    e_2=e_1;
    e_1=e;
    u2_3=u2_2;
    u2_2=u2_1;
    u2_1=u2;
       
       
    /*   
    if (e2>10||e2<-10)
    {
    //u4_t=Kp*(e2-0.809*e2_1)-0.3*u4_1+0.1*u4_2; //Kp=1.69
    u4_t=Kp*(e2-1.35*e2_1+0.38*e2_2)-0.11*u4_1+0.0726*u4_2;
    }
    else u4_t=0;
    if(u4_t>DeadZone2||u4_t<-DeadZone2)
    u4=(int)u4_t;
    else u4=0;
    e2_2=e2_1;
    e2_1=e2;
    u4_2=u4_1;
    u4_1=u4;
    */
    
    u4=0;
    
    if(u2==0&&u4==0&&j==0)
    {
    Serial.println(EncoderCounter1,DEC);  
    Serial.println(EncoderCounter2,DEC);  
    j=1;
    }
    
    
    if(u2 < -ControlDeadzone)                                                                                                                                                                                                                                                 
    {
      digitalWrite(Enable1, LOW);
      digitalWrite(Enable2, HIGH);
      u2 = abs(u2) + DeadZone; 
    }
    else
    if(u2 > ControlDeadzone)
    {
      digitalWrite(Enable1, HIGH);
      digitalWrite(Enable2, LOW);
      u2 = u2 + DeadZone;
    }
    else u2 = 0;
    
    if(u4 < -ControlDeadzone)
    {
      digitalWrite(Enable3, LOW);
      digitalWrite(Enable4, HIGH);
      u4 = abs(u4) + DeadZone; 
    }
    else
    if(u4 > ControlDeadzone)
    {
      digitalWrite(Enable3, HIGH);
      digitalWrite(Enable4, LOW);
      u4 = u4 + DeadZone;
    }
    else u4 = 0;
    delay(3); //you may try to add some delay here, e^(-j*delay()*s)
    
    analogWrite(PWM1,constrain(u2,0,255));
    analogWrite(PWM2,constrain(u4,0,255));
    time = micros();/
  }
}

void doEncoder1()
{ 
  if(digitalRead(encoder1PinA) == digitalRead(encoder1PinB)){
    EncoderCounter1++;
  }
  else{
    EncoderCounter1--;
  }
}

void doEncoder2()
{
  if(digitalRead(encoder2PinA) == digitalRead(encoder2PinB)){
    EncoderCounter2++;
  }
  else{
    EncoderCounter2--;
  }
}
\end{lstlisting}



\chapter*{Appendice C\\ Schema del circuito del ponte H}
\addcontentsline{toc}{chapter}{Appendice C - Schema del circuito del ponte H}

\begin{figure}[h]
\centering
\includegraphics[width=13cm]{imgs/circuito.jpg}
\caption{Schema del circuito di controllo dei motori}
\label{fig:circuito}
\end{figure}
